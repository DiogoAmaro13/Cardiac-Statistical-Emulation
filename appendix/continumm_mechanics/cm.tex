\documentclass[a4paper,11pt]{article}
\usepackage[utf8]{inputenc}
\usepackage[T1]{fontenc}
\usepackage{amsmath, amssymb, amsfonts}
\usepackage{graphicx}
\usepackage{float}
\usepackage{booktabs}
\usepackage{geometry}
\usepackage{caption}
\usepackage{subcaption}
\usepackage{siunitx}
\usepackage{hyperref}

\geometry{margin=2.5cm}
\graphicspath{{images/}}
\title{Experimental Design for Left Ventricular Biomechanics}
\author{Diogo Amaro}
\date{\today}

\begin{document}
\maketitle

\section{Experimental Design}
\label{subsec:expdes}
To ensure thorough coverage of the input parameter space, it is important to design a framework capable of exploring every possible combination as effectively as possible. Experimental design refers to the systematic planning of physical experiments to efficiently explore a parameter space and extract meaningful insights while minimizing both number of simulations and time. Basically, a good experimental design should aim to minimize the number of runs needed to acquire as much information as possible~\cite{fang2005design}. To achieve this, various techniques have been developed and applied, four of which were initially considered and are briefly described below.

\end{document}